\documentclass[12pt]{article}

%% preamble: Keep it clean; only include those you need
\usepackage{amsmath}
\usepackage[margin = 1in]{geometry}
\usepackage{graphicx}
\usepackage{booktabs}
\usepackage{natbib}

% for space filling
\usepackage{lipsum}
% highlighting hyper links
\usepackage[colorlinks=true, citecolor=blue]{hyperref}


%% meta data

\title{Proposal: Block Bootstrap for the Kolmogorov-Smirnov Test}
\author{Mathew Chandy\\
  Department of Statistics\\
  University of Connecticut
}

\begin{document}
\maketitle


\paragraph{Introduction}
%Background about your research; touch the three questions to be addressed in an
%introduction; cite relevant refernces~\citep[e.g.,][]{dwivedi2017analysis}.
The Kolmogorov-Smirnov (KS) test is a useful goodness-of-fit statistic. It 
can be used to see if a population follows a hypothesized distribution and has 
and can be applied to a variety of fields. It has
been used to analyze the random distribution of cosmic microwave background 
radiation \citep{naess2012application}. A common misuse of the KS test is when
the hypothesized distribution has unspecified parameters. 
\citet{babu2004goodness} approaches this scenario using basic 
non-parametric bootstrap. There is also the scenario where both the hypothesized 
distribution has unspecified parameters and additionally the data are 
serially dependent. \citet{zeimbekakis2022misuses} demonstrates a remedy for
this scenario using semi-parametric bootstrap. This study aims to address this
scenario with non-parametric block bootstrap.




\paragraph{Specific Aims}
%Formulate your research question;
%translate your research question into statistical/data science questions
Can the Kolmogorov-Smirnov test using non-parametric block bootstrap
hold its size under the null hypothesis and is it powerful under the alternate 
hypothesis, even when the data is serially dependent?



\paragraph{Data}
%Hopefully, you have identified the data needed for your project. Give a
%description about it.
The data for this project will be generated in R. 
First, we must show that the method works under the null hypothesis. In this 
case for example, we would use the method on data generated from a simulation
of an AR(1) process with an underlying marginal $N(8,8)$ distribution given the 
null hypothesis that the data is normally distributed. We could do a similar 
procedure where the underlying marginal distribution and
the null distribution are both $\Gamma(8,1)$
Then, we must show that the method works when the null hypothesis is false.
In this case for example, we would use a similar procedure where the underlying
marginal distribution is $N(8,8)$ and the null distribution is $\Gamma(8,1)$,
or vice versa.



\paragraph{Research Design and Methods}
%What design or methods will you use?
%Cite relevant references~\citep[e.g.,][]{wild2004global}.
In the case that the null hypothesis is true:
we will approximate the KS statistic for a sample using 
block bootstrap with $B$ bootstrap samples. We will replicate this (new 
generated sample + block bootstrap KS test procedure) 1000 times, and assess 
if the p-values for all the tests are uniformally distributed. If this is the 
case, we can say
that the method works under the null hypothesis.
In the case that the null hypothesis is false:
we will approximate the KS statistic for a sample using 
block bootstrap with $B$ bootstrap samples. We will replicate this (new 
generated sample + block bootstrap KS test procedure) 1000 times, and assess
how reliably the test rejects the null hypothesis (ideally this should be 
100\% of the time).




\paragraph{Discussion}
%What are the most challenge parts of the task?
%What are the limitations of your work? What is your fall-back plan if
%something unexpected happens?
The most challenging part of this task will be finding the sample size at which
the method reliably works when the null is true versus when the null is false.
This work maybe limited by some general drawbacks of block 
bootstrap in terms of how effectively it imitates the data-generating 
process, as noted by \citet{buhlmann2002bootstraps}. Because the data for this
study is simulated through R, if something unexpected happens, it is not the end
of the world because I can just check my code. If after having verified my 
code, I find that the same unexpected thing is happening, it is an interesting 
finding that I can include in the paper/





\bibliography{../manuscript/citations}
\bibliographystyle{chicago}

\end{document}